\documentclass[11pt,a4j]{article}
\usepackage[dvipdfmx]{graphicx,color}
\usepackage{wrapfig}
\setlength{\topmargin}{-1.5cm}
\setlength{\textwidth}{15.5cm}
\setlength{\textheight}{25.2cm}
\newlength{\minitwocolumn}
\setlength{\minitwocolumn}{0.5\textwidth}
\addtolength{\minitwocolumn}{-\columnsep}
%\addtolength{\baselineskip}{-0.1\baselineskip}
%
\def\Mmaru#1{{\ooalign{\hfil#1\/\hfil\crcr
\raise.167ex\hbox{\mathhexbox 20D}}}}
%
\begin{document}
\newcommand{\fat}[1]{\mbox{\boldmath $#1$}}
\newcommand{\D}{\partial}
\newcommand{\w}{\omega}
\newcommand{\ga}{\alpha}
\newcommand{\gb}{\beta}
\newcommand{\gx}{\xi}
\newcommand{\gz}{\zeta}
\newcommand{\vhat}[1]{\hat{\fat{#1}}}
\newcommand{\spc}{\vspace{0.7\baselineskip}}
\newcommand{\halfspc}{\vspace{0.3\baselineskip}}
\bibliographystyle{unsrt}
%\pagestyle{empty}
\newcommand{\twofig}[2]
 {
   \begin{figure}[h]
     \begin{minipage}[t]{\minitwocolumn}
         \begin{center}   #1
         \end{center}
     \end{minipage}
         \hspace{\columnsep}
     \begin{minipage}[t]{\minitwocolumn}
         \begin{center} #2
         \end{center}
     \end{minipage}
   \end{figure}
 }
%%%%%%%%%%%%%%%%%%%%%%%%%%%%%%%%%
%\vspace*{\baselineskip}
\begin{center}
{\Large \bf A Synthetic Aperture Focusing Algorithm for Ultrasonic Imaging}
\end{center}
\vspace{10mm}
%%%%%%%%%%%%%%%%%%%%%%%%%%%%%%%%%%%%%%%%%%%%%%%%%%%%%%%%%%%%%%%%
\section{Imaging Problem}
\hspace{\parindent}
We develop an ultrasonic imaging method for a flaw in a plate.
Synthetic aperture focusing concept will be used for this purpose. 
The imaging algorithm is formulated for a reasonably general model of a 
ultrasonic testing (UT) shown schematically in Fig.*. 
In this model, it is assumed that the incident ultrasonic wave is 
excited at a point $\fat{x}_s$ on the top plate surface.
The echoes returning from the flaw are also observed on the top surface 
over an aperture $\cal R$.
The ultrasonic echo signal captured with an transducer is usually given as a temporal 
 waveform, and is called an A-scan in the context of ultrasonic nondestructive testing.
We adopt this terminology in this article, and denote the A-scan captured 
at $\fat{x}$ as $a(\fat{x},t;\fat{x}_s)$ where $t$ is the time elapsed from 
the time of incident wave excitation. 
Since A-scans are captured on $\cal R$, the whole dataset 
 available for the imaging is written as 
\begin{equation}
	{\cal D} = \left\{
		a(\fat{x},t;\fat{x}_s)\left| \fat{x}\in {\cal R},t>0 \right.
	\right\}.
	\label{eqn:bscan_data}
\end{equation}
For given dataset $\cal D$, we consider synthesizing an image $I(\fat{x}_P)$ 
of the unknown flaw, thereby estimate the location and size of the flaw.
%%%%%%%%%%%%%%%%%%%%%%%%%
\section{Synthetic Aperture Imaging Algorithm}
\hspace{\parindent}
The present synthetic imaging algorithm may be divided into two parts.
The first part is a delay-and-sum operation on the measured A-scans ${\cal D}$. 
The second part is a sampling or projection operation on the resulting summed A-scan.
The delay law and the nature of the samplig/projection are both related closely 
to the travel time of the ultrasonic wave.
It is thus convenient to introduce a handy notation for the travel time. 
When an ultrasonic wave travels from $\fat{y}$ to $\fat{x}$ 
 taking a flight time $t_f$, we write
\begin{equation}
	t_f=T_f(\fat{x},\fat{y}).
	\label{eqn:Tf}
\end{equation}
It reads "the travel time from \fat{y} to \fat{x} is $t_f$".
In eq.(\ref{eqn:Tf}), it is meant that $t_f$ is a single travel time, while 
$T_f$ is the function which returns the travel time for given point of 
departure and arrival. 
To process an A-scan with the help of time-of-flight concept, it is necessary to 
decompose $a(\fat{x},t;\fat{x}_s)$ into the incident and scattered components as 
\begin{equation}
	a(\fat{x},t;\fat{x}_s)=
	a^{in}(\fat{x},t;\fat{x}_s)
	+
	a^{sc}(\fat{x},t;\fat{x}_s).
	\label{eqn:wv_split}
\end{equation}
As in eq.(\ref{eqn:wv_split}), quantities associated with the incident and scattered wave 
components are and will be denoted with the superscripts "{\it in}" and "{\it sc}", respectively.
With this decomposition, we may associate a flight time
\begin{equation}
	t^{in}(\fat{x})=T_f(\fat{x},\fat{x}_s), (\fat{x} \in {\cal R})
	\label{eqn:def_tin}
\end{equation}
with the incident wave component $a^{in}(\fat{x},t;\fat{x}_s)$. 
If $t^{in}(\fat{x})$ is estimated for every $\fat{x} \in {\cal R}$, 
the incident wave pulse in each A-scan can be aligned at the origin 
of the coordinate $t=0$ by translating $a(x,t)$ by $t^{in}(\fat{x})$ as 
$a\left(\fat{x},t+t^{in}(\fat{x})\right)$.
The A-scans translated thus are then superimposed as 
\begin{equation}
	\bar{a}^{in}(t)=\int_{\cal R} a\left(\fat{x},t+t^{in}(\fat{x})\right)dx,
	\label{eqn:asum_in}
\end{equation}
so that the incident wave pulse in $\bar{a}^{in}(t)$ be amplified to a degree 
dependent on the accuracy of the estimated travel time.
Clearly, eq.(\ref{eqn:asum_in}) is a delay-and-sum in which the waveform $a(\fat{x},t)$ 
is delayed by $-t^{in}(\fat{x})$ and summed over the aperture $\cal R$.

A summed A-scan similar to eq.(\ref{eqn:asum_in}) may be obtained for the 
scattered wave component $a^{sc}(\fat{x},t)$. 
Associating the flight time, however, with the scattered wave component is not straightforward. 
This is because scattered waves are generated at every point on the scattereing object. 
Consequently, it is not possible to pinpoint where exaxtly is a scattered wave is originated from. 
 A simple and practical trick to circumvent this difficulty is to model the scattering object 
 as a collection of small, point-like scaterers distributed densely but discretely over 
 the surface of the object. Then the flight time can be assigned to each constituent point scatterer 
 with no ambiguity. This approach may be formulated more rigorously if we descritize an 
 integral representaion of the scattered field, which is beyond the scope of this article.
When one of the point scatteres is located at $\fat{x}_P$, the flight time for the scattered wave 
 is given by 
\begin{equation}
	t_P^{sc}(\fat{x})=
	T_f(\fat{x}_P,\fat{x}_s)
	+
	T_f(\fat{x},\fat{x}_P).
	\label{eqn:tsc}
\end{equation}
The subscript $P$ in eq.(\ref{eqn:tsc}) is to remind that the flight time depend 
on the location of the point scatterer.
Then we can align and amplify the scattered wave packets at $t=0$ by the following delay-and-sum operation. 
\begin{equation}
	\bar{a}^{sc}_P(t)=\int _{\cal R} a^{sc}(\fat{x},t+t^{sc}_P(\fat{x}))d\fat{x}
	\label{eqn:asum_sc}
\end{equation}
The degree of amplicification depends obviously on the accuracy of the estimated $t^{sc}_P$. 
It is equally or more important to note that the amplification would not occur 
if there were not the scattere at $\fat{x}_P$, or if the scattered waves from $\fat{x}_P$ did 
not take the path we have considerd in estimating $t^{sc}_P(\fat{x})$
On the other hand, if there is no scattering object at $\fat{x}_P$, then there is no wave packet in $a^{sc}(x,t)$ 
to be amplified. Thus the summed A-scan would be rather quiet having a small norm.
This means the summed A-scan $\bar a^{sc}_P$ can be used as a measure of scattering intensity at $\fat{x}_P$. 
If the intensity is measued by a scalar quantitiy by samping or a projection, then we may synthesize a 
map of scattered intesity by 
\begin{equation}
	%I(\fat{x}_P)={\cal S} \left\{ \bar{a}^{sc}_P(t) \right\}
	I(\fat{x}_P)={\cal S} \left[ \bar{a}^{sc}_P(t) \right]
	\label{eqn:}
\end{equation}
where ${\cal S}$ denote the sampling(projection) operator from a space of 
 A-scan waveforms to the non-negative real coordinate $R^{+}$.
 The simplest choice for $\cal S$ is a sampling by Dirac's delta function 
\begin{equation}
	{\cal S} \left[ 
		\bar{a}^{sc}_P(t) 
		\right]
	=\left<
	\delta(t),\, \bar{a}^{sc}_P
	\right>
	=
	\bar{a}^{sc}_P(0)
\end{equation}
where $\left<\cdot,\, \cdot \right>$ means the dot product defined by 
\begin{equation}
	\left< f,g \right>=\int f(t)g(t) dt.
	\label{eqn:}
\end{equation}
This is reasonable because the $\bar{a}^{sc}_P$ is constructed so that the 
wave packet is amplified at $t=0$. However, this is not often a good choice 
as the real A-scan is not an ideal pulse, and hence the peak amplitude 
always comes after the theoretically estimated flight time. 
A better alternative to automatically account such delay is to take a cross correlation 
\begin{equation}
	{\cal S} \left[ 
		\bar{a}^{sc}_P(t) 
		\right]
	=
	\frac{
		\left< 
			\bar{a}^{in}(t) 
			,\bar{a}^{sc}_P(t) 
		\right>
	}{
		\left<
			\bar{a}^{in}(t) 
			,
			\bar{a}^{in}(t) 
		\right>
	}
	\label{eqn:S_cor}
\end{equation}
where the uncertainty regarding the delay is compensated by the 
uncertainty of summed incident wave pulse shape.
We can further generalize the projection $\cal S$, e.g. as 
\begin{equation}
	{\cal S} \left[ 
		\bar{a}^{sc}_P(t) 
		\right]
	=
	\left\|a^{in}(t)*a^{sc}(t)\right\|_p
	\label{eqn:}
\end{equation}
where $*$ denotes a convolution product and $\left\| \cdot \right\|_p$ means the 
$p-$norm. However, we're not going to elaborate the projection, but rather, investigate 
how a relatively simple projection (\ref{eqn:S_cor}) works with the A-scans measured with an LDV. 
\section{Numerical Implementation}
\subsection{Incident and scattered wave splitting}
\begin{equation}
	u(x,t)=\int U(k,\omega) e^{i(kx-\omega t)} dkd\omega
	\label{eqn:}
\end{equation}
The spectral content $U(k,\omega)$ corresponds to the 
left(right) going wave in space-time domain when $k\omega >0 (<0)$. 
Thus we can split $u$ into the left and right going wave by 
a frequency-wave number domain filtering 
\begin{equation}
	u^{\pm}(x,t)=\int U(k,\omega)H(\pm k\omega) e^{i(kx-\omega t)} dkd\omega
	\label{eqn:}
\end{equation}
where $H(s)$ is the Heaviside unit step function.
\begin{equation}
	H(s)=\left\{
	\begin{array}{cc}
		1 & (s >0) \\
		0 & (s <0)
	\end{array}
	\right.
	\label{eqn:}
\end{equation}
\subsection{Construction of the flight time function}
There are infinitely many wave paths in the plate due to the wave reverberation.
The number of paths increases exponentially if the mode conversion upon reflection 
at the plate surfaces are taken into account.
We look at small number of them to make the problem tractable.
Furthermore, mode preseving ray paths that accampanies small number of reflections 
 will be considerd. 
 This makes the ray tracing problem greatly simple. 
 To obtain the travel time, in a homogeneous material it suffice to 
 obtain the travel path that connect the source to the receiver point 
 after a given number of reflection.
 The ray paths for an arbitray many reflections can be constructed by 
 a successive application of a so called "image point method".
The procedure of the image point method is illustrated in Fig.*, which should make 
the concept of the method clear.
\end{document}

