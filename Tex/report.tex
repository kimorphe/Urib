\documentclass[11pt,a4j]{article}
\usepackage[dvipdfmx]{graphicx,color}
\usepackage{wrapfig}
\setlength{\topmargin}{-1.5cm}
\setlength{\textwidth}{15.5cm}
\setlength{\textheight}{25.2cm}
\newlength{\minitwocolumn}
\setlength{\minitwocolumn}{0.5\textwidth}
\addtolength{\minitwocolumn}{-\columnsep}
%\addtolength{\baselineskip}{-0.1\baselineskip}
%
\def\Mmaru#1{{\ooalign{\hfil#1\/\hfil\crcr
\raise.167ex\hbox{\mathhexbox 20D}}}}
%
\begin{document}
\newcommand{\fat}[1]{\mbox{\boldmath $#1$}}
\newcommand{\D}{\partial}
\newcommand{\w}{\omega}
\newcommand{\ga}{\alpha}
\newcommand{\gb}{\beta}
\newcommand{\gx}{\xi}
\newcommand{\gz}{\zeta}
\newcommand{\vhat}[1]{\hat{\fat{#1}}}
\newcommand{\spc}{\vspace{0.7\baselineskip}}
\newcommand{\halfspc}{\vspace{0.3\baselineskip}}
\bibliographystyle{unsrt}
%\pagestyle{empty}
\newcommand{\twofig}[2]
 {
   \begin{figure}[h]
     \begin{minipage}[t]{\minitwocolumn}
         \begin{center}   #1
         \end{center}
     \end{minipage}
         \hspace{\columnsep}
     \begin{minipage}[t]{\minitwocolumn}
         \begin{center} #2
         \end{center}
     \end{minipage}
   \end{figure}
 }
%%%%%%%%%%%%%%%%%%%%%%%%%%%%%%%%%
%\vspace*{\baselineskip}
\begin{center}
{\Large \bf A Synthetic Aperture Focusing Algorithm for Ultrasonic Imaging}
\end{center}
\vspace{10mm}
%%%%%%%%%%%%%%%%%%%%%%%%%%%%%%%%%%%%%%%%%%%%%%%%%%%%%%%%%%%%%%%%
Let $a(\fat{x},t;\fat{x}_s)$ be an A-scan waveform of time $t$
measured at $\fat{x}$ when the incident wave is excited by a 
 source point at $\fat{x}_s$. 
Suppose that a set of A-scans 
\begin{equation}
	{\cal D} = \left\{
		a(\fat{x},t;\fat{x}_s)\left| \fat{x}\in {\cal R},t>0 \right.
	\right\}
	\label{eqn:bscan_data}
\end{equation}
are measured over a receiver aperture ${\cal R}$, and consider synthesizing an 
image of the scattering objects from the given dataset ${\cal D}$.
A synthetic apertue algorithm is used for this purpose.
In developing the synthetic aperture imaging method, it is assumed that the waveform 
$a(\fat{x},t;\fat{x}_s)$ can be decomposed into the incident and 
scattered components as 
\begin{equation}
	a(\fat{x},t;\fat{x}_s)=
	a^{in}(\fat{x},t;\fat{x}_s)
	+
	a^{sc}(\fat{x},t;\fat{x}_s).
	\label{eqn:wv_split}
\end{equation}
As in eq.(\ref{eqn:wv_split}), quantities associated with the incident and scattered wave 
components are and will be denoted by the superscripts "in" and "sc", respectively.

The imaging algorithm is divided into two parts.
The first part is the delay-and-sum operation on the A-scans in ${\cal D}$. 
The second part is the sampling of a wave amplitude from the summed A-scan.
The laws of the delay-and-sum and that of samplig are both based on the flight time.
It is hence convenient to introduce a time-of-flight function $T_f$.
The function $T_f(\fat{x},\fat{y})$ returns a time $\tau$ required for the ultrasonic 
wave to travel from a source ($\fat{y}$) to an obsevation point($\fat{x}$), and written as 
\begin{equation}
	\tau=T_f(\fat{x},\fat{y}).
	\label{eqn:Tf}
\end{equation}
If $T_f(\fat{x},\fat{y})$ is considered as a fuction of $\fat{x}$, it reduces to a travel time curve. 
Returning to the decomposition of eq.(\ref{eqn:wv_split}), it is natural to define 
the fligt time of the incident wave as 
\begin{equation}
	t^{in}(\fat{x})=T_f(\fat{x},\fat{x}_s), (\fat{x} \in {\cal R}).
	\label{eqn:def_tin}
\end{equation}
If $t^{in}(\fat{x})$ is given for $\fat{x} \in {\cal R}$, the incident wave pulse 
in each A-scan can be aligned at the origin of coordinate $t=0$ by translating 
$a(x,t)$ by $T(x,x_s)$ as $a(x,t+T(x,x_s))$.
If the translated A-scans are superimposed as 
\begin{equation}
	\bar{a}^{in}(t)=\int_{\cal R} a(x,t+T(x,x_s))dx,
	\label{eqn:asum_in}
\end{equation}
then the incient wave packets are interfered constructively and amplified in $\bar{a}^{in}(t)$ at $t=0$.
Equation(\ref{eqn:asum_in}) is an exmaple of delay and sum operation, in which the waveform $a(x,t)$ 
is delayed by $-T_f(x,x_s)$ and summed over $\cal R$.
The summed A-scan similar to eq.(\ref{eqn:asum_in}) can be defined for the scattered waveform $a^{sc}(x,t)$. 
However, defining the flight time for the scattered wave component is not straightforward. 
For the scattering object of a general shape, it is not possible to define a unique flight time as every
 part of the scattering object could generate a scattered waves. 
 A simple trick to circumvent this difficulty is to consider the scattering objct as a  collection of 
 small, point-like scaterer distributed over the surface of the object. 
 Then the scatted wave flight time can be clearly defined for the constitutnet point scatteres. 
 This approach may be formulated more rigorously if an integral representaion of the scattered field, 
 which is beyon the scope of this article.
When a point scattere is located at a pixel point $\fat{x}_P$, the scattered wave flight time  
 is given by 
\begin{equation}
	t_P^{sc}(\fat{x})=
	T_f(\fat{x}_P,\fat{x}_s)
	+
	T_f(\fat{x},\fat{x}_P)
	\label{eqn:}
\end{equation}
where the subscript $P$ is there as a remider that the flight time depend on the pixel point $\fat{x}_P$.
Once $\fat{x}_P$ is specified, it is ready to align the scattered wave packet at 
the origin of the coordinate and sum them up so that the wave packets be amplyfied in the 
summed A-scan. The delay-and-sum operation may be written as follows. 
\begin{equation}
	\bar{a}^{sc}_P(t)=\int _{\cal R} a^{sc}(\fat{x},t-t^{sc}_P(\fat{x}))d\fat{x}
	\label{eqn:asum_sc}
\end{equation}
On the other hand, if there is no scattering object at $\fat{x}_P$, then there is no wave packet in $a^{sc}(x,t)$ 
to be amplified. Thus the summed A-scan would be rather quiet having a small norm.
This means the summed A-scan $\bar a^{sc}_P$ can be used as a measure of scattering intensity at $\fat{x}_P$. 
If the intensity is measued by a scalar quantitiy by samping or a projection, then we may synthesize a 
map of scattered intesity by 
\begin{equation}
	%I(\fat{x}_P)={\cal S} \left\{ \bar{a}^{sc}_P(t) \right\}
	I(\fat{x}_P)={\cal S} \left[ \bar{a}^{sc}_P(t) \right]
	\label{eqn:}
\end{equation}
where ${\cal S}$ denote the sampling(projection) operator from a space of 
 A-scan waveforms to the non-negative real coordinate $R^{+}$.
 The simplest choice for $\cal S$ is a sampling by Dirac's delta function 
\begin{equation}
	{\cal S} \left[ 
		\bar{a}^{sc}_P(t) 
		\right]
	=\left<
	\delta(t),\, \bar{a}^{sc}_P
	\right>
	=
	\bar{a}^{sc}_P(0)
\end{equation}
where $\left<\cdot,\, \cdot \right>$ means the dot product defined by 
\begin{equation}
	\left< f,g \right>=\int f(t)g(t) dt.
	\label{eqn:}
\end{equation}
This is reasonable because the $\bar{a}^{sc}_P$ is constructed so that the 
wave packet is amplified at $t=0$. However, this is not often a good choice 
as the real A-scan is not an ideal pulse, and hence the peak amplitude 
always comes after the theoretically estimated flight time. 
A better alternative to automatically account such delay is to take a cross correlation 
\begin{equation}
	{\cal S} \left[ 
		\bar{a}^{sc}_P(t) 
		\right]
	=
	\frac{
		\left< 
			\bar{a}^{in}(t) 
			,\bar{a}^{sc}_P(t) 
		\right>
	}{
		\left<
			\bar{a}^{in}(t) 
			,
			\bar{a}^{in}(t) 
		\right>
	}
	\label{eqn:S_cor}
\end{equation}
where the uncertainty regarding the delay is compensated by the 
uncertainty of summed incident wave pulse shape.
We can further generalize the projection $\cal S$, e.g. as 
\begin{equation}
	{\cal S} \left[ 
		\bar{a}^{sc}_P(t) 
		\right]
	=
	\left\|a^{in}(t)*a^{sc}(t)\right\|_p
	\label{eqn:}
\end{equation}
where $*$ denotes a convolution product and $\left\| \cdot \right\|_p$ means the 
$p-$norm. However, we're not going to elaborate the projection, but rather, investigate 
how a relatively simple projection (\ref{eqn:S_cor}) works with the A-scans measured with an LDV. 
\end{document}
