\documentclass[11pt,a4j]{article}
\usepackage[dvipdfmx]{graphicx,color}
\usepackage{wrapfig}
\setlength{\topmargin}{-1.5cm}
\setlength{\textwidth}{15.5cm}
\setlength{\textheight}{25.2cm}
\newlength{\minitwocolumn}
\setlength{\minitwocolumn}{0.5\textwidth}
\addtolength{\minitwocolumn}{-\columnsep}
%\addtolength{\baselineskip}{-0.1\baselineskip}
%
\def\Mmaru#1{{\ooalign{\hfil#1\/\hfil\crcr
\raise.167ex\hbox{\mathhexbox 20D}}}}
%
\begin{document}
\newcommand{\fat}[1]{\mbox{\boldmath $#1$}}
\newcommand{\D}{\partial}
\newcommand{\w}{\omega}
\newcommand{\ga}{\alpha}
\newcommand{\gb}{\beta}
\newcommand{\gx}{\xi}
\newcommand{\gz}{\zeta}
\newcommand{\vhat}[1]{\hat{\fat{#1}}}
\newcommand{\spc}{\vspace{0.7\baselineskip}}
\newcommand{\halfspc}{\vspace{0.3\baselineskip}}
\bibliographystyle{unsrt}
%\pagestyle{empty}
\newcommand{\twofig}[2]
 {
   \begin{figure}[h]
     \begin{minipage}[t]{\minitwocolumn}
         \begin{center}   #1
         \end{center}
     \end{minipage}
         \hspace{\columnsep}
     \begin{minipage}[t]{\minitwocolumn}
         \begin{center} #2
         \end{center}
     \end{minipage}
   \end{figure}
 }
%%%%%%%%%%%%%%%%%%%%%%%%%%%%%%%%%
%\vspace*{\baselineskip}
\begin{center}
{\Large \bf A Synthetic Aperture Focusing Method for Ultrasonic Imaging}
\end{center}
\vspace{10mm}
%%%%%%%%%%%%%%%%%%%%%%%%%%%%%%%%%%%%%%%%%%%%%%%%%%%%%%%%%%%%%%%%
Let $a(\fat{x},t;\fat{x}_s)$ be an A-scan waveform of time $t$
measured at $\fat{x}$ when the incident wave is excited by a 
 source located at $\fat{x}_s$. 
 We consider synthesizing an ultrasonic image of scattering object for 
 given set of waveforms $\cal D$ such that 
\begin{equation}
	{\cal D} = \left\{
		a(\fat{x},t;\fat{x}_s)\left| \fat{x}\in {\cal R},t>0 \right.
	\right\}
	\label{eqn:bscan_data}
\end{equation}
where $\cal R$ denotes the measurement aperture.
It is assumed that the waveform $a(\fat{x},t;\fat{x}_s)$ may be decomposed into 
the incident and scattered components as 
\begin{equation}
	a(\fat{x},t;\fat{x}_s)=
	a^{in}(\fat{x},t;\fat{x}_s)
	+
	a^{sc}(\fat{x},t;\fat{x}_s).
	\label{eqn:wv_split}
\end{equation}
In what follows, quantities associated with the incident and scattered wave 
components are denoted by the superscripts "in" and "sc", respectively.
To explain the delay-and-sum operation we introduce the time-of-flight 
function:
\begin{equation}
	t=T_f(\fat{x},\fat{y})
	\label{eqn:Tf}
\end{equation}
where $\fat{x}$ and $\fat{y}$ denote the points of observation and transmission 
,respectively. 
The time-of-flight function $T_f$ is a fuction that returns the 
travel time required for a wave excited at $\fat{y}$ to arrive at $\fat{x}$.
The incident wave 
\begin{equation}
	t^{in}(\fat{x})=T_f(\fat{x},\fat{x}_s)
	\label{eqn:}
\end{equation}
\begin{equation}
	t_P^{sc}(\fat{x})=
	T_f(\fat{x}_P,\fat{x}_s)
	+
	T_f(\fat{x},\fat{x}_P)
	\label{eqn:}
\end{equation}
\begin{equation}
	\bar{a}^{in}(t)=\int _{\cal R} a^{in}(\fat{x},t-t^{in}(\fat{x})+t^{in}(\fat{x}_R))d\fat{x}
	\label{eqn:}
\end{equation}
\begin{equation}
	\bar{a}^{sc}_P(t)=\int _{\cal R} a^{sc}(\fat{x},t-t^{sc}_P(\fat{x})+t_P^{sc}(\fat{x}_R))d\fat{x}
	\label{eqn:}
\end{equation}
\begin{equation}
	%I(\fat{x}_P)={\cal S} \left\{ \bar{a}^{sc}_P(t) \right\}
	I(\fat{x}_P)={\cal S} \left[ \bar{a}^{sc}_P(t) \right]
	\label{eqn:}
\end{equation}
\begin{equation}
	{\cal S} \left[ 
		\bar{a}^{sc}_P(t) 
		\right]
	=
	\bar{a}^{sc}_P(t^{sc}_P) 
	\left(
	=\left<
	\delta(t-t_P^{sc}),\,
	\bar{a}^{sc}_P
	\right>
	\right)
\end{equation}
\begin{equation}
	\left< f,g \right>=\int f(t)g(t) dt
	\label{eqn:}
\end{equation}
\begin{equation}
	{\cal S} \left[ 
		\bar{a}^{sc}_P(t) 
		\right]
	=
	\left< 
		\bar{a}^{in}(t+t^{in}-t^{sc}_P) 
		,\bar{a}^{sc}_P(t) 
	\right>
\end{equation}


In the followings, the incident wave component is used as a rerference waveform, 
while the scattered wave component as a signal to synthesize an image of 
scattering object.
The imaging alogorithm is consisted of two steps. The first step is 
the delay-and-sum operation on the A-scans to amplifiy the scattered wave packet of interest.
The second step is a sampling operation on the resulting A-scan to assign an image 
intensity to a particular pixel. We go through those steps as follows. 

If the phase velocity $c$ of ultasonic wave is given, the time-of-flight 
required for the wave to travel from a given source $\fat{x}_s$ to an observation point 
$\fat{x}$ can be caluculated theoretically. 
Let the theoretically obtained time-of-flight be
To 

\end{document}
